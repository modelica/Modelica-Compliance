\documentclass{article}
\usepackage[a4paper]{geometry}
\usepackage[utf8]{inputenc}
\usepackage[T1]{fontenc}
\usepackage{listings}
\usepackage{lmodern}
\usepackage{color}
\usepackage{calc}

\begin{document}
\definecolor{keywordcolor}{rgb}{0,0,.4}
\definecolor{literalcolor}{rgb}{.90,0,0}
\definecolor{stringcolor}{rgb}{0.133,0.545,0.133}
\definecolor{bgcolor}{rgb}{0.95,0.95,0.95}
\definecolor{typecolor}{rgb}{0.60,0.2,0.2}

\newcommand{\textcolordummy}[2]{#2}

\lstdefinelanguage{modelica}
{
  morekeywords=[1]{
    algorithm,and,annotation,as,assert,block,break,case,class,connect,connector,
    constant,constrainedby,der,discrete,each,else,elseif,elsewhen,encapsulated,
    end,enumeration,equality,equation,expandable,extends,external,failure,final,
    flow,for,function,guard,if,import,in,initial,inner,input,List,local,loop,
    match,matchcontinue,model,not,operator,Option,or,outer,output,package,parameter,
    partial,protected,public,record,redeclare,replaceable,return,stream,
    subtypeof,then,Tuple,type,uniontype,when,while},
  morekeywords=[2]{true, false},
  morekeywords=[3]{Real,Integer,String,Boolean},
  % Do not make true,false keywords because fn(true,x, false ) shows up as fn(true,x, *false*)
  sensitive=true,
  comment=[l]//,
  morecomment=[s]{/*}{*/},
  alsodigit={.,-},
  morestring=[b]',
  morestring=[b]",
  literate={0}{{\textcolor{literalcolor}{0}}}{1}%
           {1}{{\textcolor{literalcolor}{1}}}{1}%
           {2}{{\textcolor{literalcolor}{2}}}{1}%
           {3}{{\textcolor{literalcolor}{3}}}{1}%
           {4}{{\textcolor{literalcolor}{4}}}{1}%
           {5}{{\textcolor{literalcolor}{5}}}{1}%
           {6}{{\textcolor{literalcolor}{6}}}{1}%
           {7}{{\textcolor{literalcolor}{7}}}{1}%
           {8}{{\textcolor{literalcolor}{8}}}{1}%
           {9}{{\textcolor{literalcolor}{9}}}{1}%
           {.0}{{\textcolor{literalcolor}{.0}}}{2}% Following is to ensure that only periods
           {.1}{{\textcolor{literalcolor}{.1}}}{2}% followed by a digit are changed.
           {.2}{{\textcolor{literalcolor}{.2}}}{2}%
           {.3}{{\textcolor{literalcolor}{.3}}}{2}%
           {.4}{{\textcolor{literalcolor}{.4}}}{2}%
           {.5}{{\textcolor{literalcolor}{.5}}}{2}%
           {.6}{{\textcolor{literalcolor}{.6}}}{2}%
           {.7}{{\textcolor{literalcolor}{.7}}}{2}%
           {.8}{{\textcolor{literalcolor}{.8}}}{2}%
           {.9}{{\textcolor{literalcolor}{.9}}}{2}%
}[keywords,comments,strings]

\lstset{
  breaklines=true, 
  language=modelica,
  basicstyle=\ttfamily\small,
  keywordstyle=[1]\color{keywordcolor}\bfseries,
  keywordstyle=[2]\color{literalcolor},
  keywordstyle=[3]\color{typecolor}\bfseries,
  stringstyle=\color{stringcolor}\let\textcolor\textcolordummy,
  backgroundcolor=\color{bgcolor},
  framexleftmargin=5pt,
  xleftmargin=5pt,
  xrightmargin=5pt,
  showstringspaces=false,
}


\title{Modelica Compliance Library Guide}
\maketitle

\section{Test case structure}
Each test case should consist of a separate file that contains one
model marked as a test case. The test case model does not need to be the
top-level class in the file, so a file can e.g. have a package with multiple
classes defined in it, but each file should only contain one test case model.
Models are marked as test cases by the test case annotation:
\begin{lstlisting}[language=modelica]
annotation(__ModelicaAssociation(TestCase(shouldPass=true|false)))
\end{lstlisting}
The shouldPass property defines whether the test case is expected to succeed or
fail. Each test case should also extend the Icons.TestCase model which supplies
a graphical icon annotation, a Documentation annotation with HTML documentation,
and an experiment annotation for simulation properties. Given below is an
example of a test case which is expected to succeed:
\begin{lstlisting}[language=modelica]
model SimpleDeclaration
  extends Icons.TestCase;
  Real x = 3;
  Real y = x;
  annotation(
    __ModelicaAssociation(TestCase(shouldPass=true)),
    experiment(StopTime=0.01),
    Documentation(
      info="<html>Tests simple component declarations.</html>"));
end SimpleDeclaration;
\end{lstlisting}
And a test case which is expected to fail:
\begin{lstlisting}[language=modelica]
model DoubleDeclaration
  extends Icons.TestCase;
  Real x;
  Real x "Double declaration of x.";
  annotation(
    __ModelicaAssociation(TestCase(shouldPass=false)),
    experiment(StopTime=0.01),
    Documentation(
      info="<html>Tests that double declaration of elements is not allowed, according to section 4.2.</html>"));
end DoubleDeclaration;
\end{lstlisting}
\section{Library structure}
The compliance library uses a package hierarchy to divide the test cases into
suitable categories, as defined by the table below. References are for the
Modelica 3.3 specification (since I haven't seen the mythical 3.2rev2). Section
references in the table uses intervals and wildcards, where 1.2-1.4 means from
section 1.2 to 1.4 and 1.1.* means section 1.1 and its subsections. Section
references does not otherwise include subsections, so section 1.1 means
\emph{only} 1.1.
%Chapter 6
%Anything testable?
%
%Chapter 10
%- 10.2 Flexible array sizes
%
%Chapter 19
%Nothing to test?
%

\medskip
\noindent
\begin{tabular}{l l}
  \textbf{Components} \\
    \indent\textbf{Declarations}     & Section 4.2-4.4.2.1, 4.4.3. \\
    \indent\textbf{Conditional}      & Conditional components, section 4.4.5. \\
    \indent\textbf{Prefixes}         & Component prefixes, section 4.4.2.2, 4.4.4.*. \\
    \indent\textbf{Variability}      & Variability prefixes, section 3.8.*, 4.4.4. \\
    \indent\textbf{Time}             & The built-in variable time, section 3.6.7.  \\
  \\
  \textbf{Classes} \\
    \indent\textbf{Declarations} \\
    \indent\indent\textbf{Long}      & ``Long'' declarations, section 4.5, 4.5.2, 4.5.3. \\
    \indent\indent\textbf{Short}     & ``Short'' declarations, section 4.5.1. \\
    \indent\textbf{Specialized}      & Restrictions on specialized classes, section 4.6. \\
    \indent\textbf{Prefixes}         & Class prefixes, section 4.4.2.2. \\
    \indent\textbf{Balancing}        & Balance checking, section 4.7. \\
    \indent\textbf{Predefined}       & The predefined types, section 4.8-4.8.4, 4.8.8.1. \\
    \indent\textbf{Enumeration}      & Enumerations, section 4.8.5. \\
  \\
  \textbf{Scoping} \\
    \indent\textbf{MemberAccess}     & The member access operator, section 3.6.6.  \\
    \indent\textbf{Visibility}       & Public and protected elements, section 4.1. \\
    \indent\textbf{NameLookup} \\
    \indent\indent\textbf{Simple}    & Section 5.3.1. \\
    \indent\indent\textbf{Composite} & Section 5.3.2. \\
    \indent\indent\textbf{Global}    & Section 5.3.3. \\
    \indent\indent\textbf{Imports}   & Section 13.2.1.*. \\
    \indent\textbf{InnerOuter}       & Section 5.4-5.5. \\
  \\
  \textbf{Operators} \\
    \indent\textbf{Arithmetic}       & Arithmetic operators $^\wedge\ast/+-$, section 3.4. \\
    \indent\textbf{Relational}       & Relational operators $==, <>$, etc. Section 3.5 and 10.6.10. \\
    \indent\textbf{Logical}          & Logical operators \textbf{not}, \textbf{and}, \textbf{or} (section 3.5). \\
    \indent\textbf{String}           & String concatenation, section 3.6.1. \\
    \indent\textbf{Mathematical}     & Operators in 3.7.1.*, except Integer and String. \\
    \indent\textbf{Conversion}       & Conversion operators in 3.7.1, Integer and String.  \\
    \indent\textbf{Events}           & Event-related operators in 3.7.3.*. \\
    \indent\textbf{Special}          & Special purpose operators in 3.7.2.*, except connection operators. \\
    \indent\textbf{If}               & If-expressions, section 3.3, 3.6.5. \\
    \indent\textbf{Precedence}       & Precedence rules in section 3.2. \\
    \indent\textbf{Associativity}    & Associativity rules in in section 3.2. \\
    \indent\textbf{Overloading}      & Overloaded operators, chapter 14. \\

\end{tabular}

\begin{tabular}{l l}
  \textbf{Inheritance} \\
    \indent\textbf{Flattening}       & Flattening of extends, section 5.6.1, 7.1-7.1.2. \\
    \indent\textbf{Restrictions}     & Base class restrictions, section 7.1.3-7.1.4. \\
  \\
  \textbf{Modification} \\
    \indent\textbf{Flattening}       & Flattening of modifications, section 7.2-7.2.3, 7.2.5. \\
    \indent\textbf{Restrictions}     & Restrictions on modifications, section 7.2.4, 7.2.6. \\
  \\
  \textbf{Redeclare} \\
    \indent\textbf{Flattening}       & Flattening of redeclares, section 7.3, 7.3.4. \\
    \indent\textbf{ConstrainingType} & Constraining types, section 7.3.2. \\
    \indent\textbf{ClassExtends}     & Class extends, section 7.3.1. \\
    \indent\textbf{Restrictions}     & Restrictions on redeclares, 7.3.3. \\
  \\
  \textbf{Equations} \\
    \indent\textbf{Equality}         & Section 8.3.1. \\
    \indent\textbf{For}              & Section 8.3.2.*. \\
    \indent\textbf{If}               & Section 8.3.4. \\
    \indent\textbf{When}             & Section 8.3.5.*. \\
    \indent\textbf{Reinit}           & Section 8.3.6. \\
    \indent\textbf{Assert}           & Section 8.3.7. \\
    \indent\textbf{Terminate}        & Section 8.3.8. \\
    \indent\textcolor{red}{\textbf{Events?}} & Section 8.5. \\
    \indent\textcolor{red}{\textbf{Initialization?}} & Section 8.6. \\
  \\
  \textbf{Algorithms} \\
    \indent\textbf{Assignment}       & Section 11.2.1.*. \\
    \indent\textbf{For}              & Section 11.2.2.*. \\
    \indent\textbf{If}               & Section 11.2.6. \\
    \indent\textbf{When}             & Section 11.2.7.*. \\
    \indent\textbf{While}            & Section 11.2.3. \\
    \indent\textbf{Break}            & Section 11.2.4. \\
    \indent\textbf{Return}           & Section 12.1.2. \\
    \indent\textbf{Reinit}           & Section 11.2.8.1. \\
    \indent\textbf{Assert}           & Section 11.2.8.2. \\
    \indent\textbf{Terminate}        & Section 11.2.8.3. \\
  \\
  \textbf{Connections} \\
    \indent\textbf{Declarations}     & Basic connect equations, section 9.1.  \\
    \indent\textbf{Operators}        & Section 3.7.2, 15.2-15.3 \\
    \indent\textbf{Expandable}       & Expandable connectors, section 9.1.3. \\
    \indent\textbf{Stream}           & Stream connectors, chapter 15. \\
    \indent\textbf{Restrictions}     & The restrictions in section 9.3.*, 15.1. \\
    \indent\textbf{Overconstrained}  & Section 9.4. \\

\end{tabular}

\begin{tabular}{l l}
  \textbf{Arrays} \\
    \indent\textbf{Declarations}     & Array declarations, section 4.4.2, 10.1.*, 10.7. \\
    \indent\textbf{Flexible}         & Flexible arrays, section 12.4.5. \\
    \indent\textbf{Indexing}         & Array indexing, slicing, section 10.5.*, 10.6.9. \\
    \indent\textbf{Functions} \\
    \indent\indent\textbf{Size}      & ndims and size, section 10.3.1. \\
    \indent\indent\textbf{Construction} & Array construction, section 10.3.3 and 10.4.*. \\
    \indent\indent\textbf{Conversion}& Dimensionality conversion functions, section 10.3.2. \\
    \indent\indent\textbf{Reductions}& Reduction expressions from section 10.3.4.*. \\
    \indent\indent\textbf{Algebra}   & Matrix and vector algebra functions, section 10.3.5. \\
    \indent\textbf{Operations} \\
    \indent\indent\textbf{Equality}  & Array equality and assignment, section 10.6.1. \\
    \indent\indent\textbf{Arithmetic}& Arithmetic operators, section 10.6.2-10.6.3, 10.6.5-10.6.7. \\
    \indent\indent\textbf{MatrixProduct}& Matrix multiplication, section 10.6.4, 10.6.8.  \\
    \indent\indent\textbf{Relational}& Relational operators, section 10.6.10. \\
    \indent\indent\textbf{Logical}   & Logical operators, section 10.6.11. \\
  \\
  \textbf{Functions} \\
    \indent\textbf{Declarations}     & Function declarations, section 12.1-12.1.1, 12.1.3. \\
    \indent\textbf{Restrictions}     & Function restrictions, section 12.2. \\
    \indent\textbf{Calls}            & Function calls, section 12.4.1, 12.4.3-12.4.4, 12.4.7. \\
    \indent\indent\textbf{Vectorization}& Vectorization of scalar functions, section 12.4.6. \\
    \indent\textbf{HigherOrder}      & Higher order functions, 12.4.2.*. \\
    \indent\textbf{Records}          & Record constructor functions, section 12.6. \\
    \indent\textbf{External}         & External functions, section 12.9.*. \\
    \indent\textbf{Derivative}       & Function derivatives, section 12.7.* \\
    \indent\textbf{Inverse}          & Function inverses, section 12.8. \\
  \\
  \textbf{Packages}                  & Chapter 13, except imports. \\
  \textbf{Annotations}               & Chapter 18. \\

\end{tabular}
\newpage
\section{Style guide}
The test cases should follow the style of the specification and MSL to the
extent that a consistent style is used. This means that class names should be
UpperCamelCase, while function and component names should be lowerCamelCase.
The contents of classes and control statements should be indented with two
spaces, and tabs should not be used.
\begin{lstlisting}[language=modelica]
package TestPackage
  type TestType = Real;

  model TestModel
    TestType testComponent;
  end TestModel;
end TestPackage;
\end{lstlisting}
Control statements should be written as in the specification with the beginning
and end parts on separate lines and the content indented.
\begin{lstlisting}[language=modelica]
for i in 1:3 loop
  if i == 1 then
    x[i] = 1;
  elseif i == 2 then
    x[i] = 2;
  else
    x[i] = 3;
  end if;
end loop;
\end{lstlisting}
Expressions and operators should usually be separated with spaces. Notable
exceptions are brackets and the range operator.
\begin{lstlisting}[language=modelica]
model Test
  Real x = 2.0;
  Real y, z;
  Integer u[1, 3] = {{1, 2, 3}};
equation
  y = x * (2 + z);
  z = sum(u[1, i] for i in 1:size(u, 1));
end Test;
\end{lstlisting}
\end{document}
